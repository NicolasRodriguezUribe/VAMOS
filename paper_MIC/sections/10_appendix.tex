\section{Additional details}
\label{sec:appendix}

\subsection{Operator portfolio}

\paragraph{Design rationale.}
The five arms were chosen to span three axes of variation:
(i) \emph{crossover geometry}---SBX and BLX-$\alpha$ operate on two parents and produce offspring along the line between them, whereas PCX, UNDX, and Simplex use three or more parents to construct offspring in a subspace defined by the parent cloud, enabling broader exploration of the decision space;
(ii) \emph{mutation tail weight}---polynomial mutation produces moderate perturbations, Gaussian mutation adds symmetric noise, Cauchy mutation occasionally injects large jumps that can escape local basins, and uniform reset resamples a variable entirely from its domain; and
(iii) \emph{conservatism}---Arm~1 (SBX+PM) is the standard default in the MOEA literature, while Arms~3--5 introduce progressively more disruptive variation.
This layered design ensures that at least one arm is likely to be effective at any stage of search, while keeping the portfolio small enough ($K{=}5$) for the bandit policy to learn meaningful reward estimates within a typical evaluation budget.

\begin{table}[t]
\centering
\scriptsize
\caption{Operator portfolio used by AOS and the random arm variant. Mutation probability is set to $1/n_{\mathrm{var}}$.}
\label{tab:portfolio}
\setlength{\tabcolsep}{4pt}
\begin{tabular}{l|l|l}
\toprule
\textbf{Arm} & \textbf{Crossover} & \textbf{Mutation} \\
\midrule
1 & SBX ($p{=}1.0$, $\eta{=}20$) & Polynomial ($p{=}1/n_{\mathrm{var}}$, $\eta{=}20$) \\
2 & PCX ($p{=}1.0$, $\sigma_\eta{=}0.1$, $\sigma_\zeta{=}0.1$) & Polynomial ($p{=}1/n_{\mathrm{var}}$, $\eta{=}20$) \\
3 & UNDX ($p{=}1.0$, $\zeta{=}0.5$, $\eta{=}0.35$) & Gaussian ($p{=}1/n_{\mathrm{var}}$, $\sigma{=}0.1$) \\
4 & Simplex ($p{=}1.0$, $\epsilon{=}0.5$) & Uniform reset ($p{=}1/n_{\mathrm{var}}$) \\
5 & BLX-$\alpha$ ($p{=}0.9$, $\alpha{=}0.5$) & Cauchy ($p{=}1/n_{\mathrm{var}}$, $\gamma{=}0.1$) \\
\bottomrule
\end{tabular}
\end{table}

\subsection{Benchmark problems}
\begin{table}[t]
\centering
\scriptsize
\caption{Summary of the 21 benchmark problems used in the experiments.}
\label{tab:problems}
\setlength{\tabcolsep}{4pt}
\begin{tabular}{l|l|r|r|l}
\toprule
\textbf{Problem} & \textbf{Description} & \textbf{Vars} & \textbf{Obj} & \textbf{Source} \\
\midrule
RE21  & Four bar truss design        & 4  & 2 & RE \\
RE24  & Welded beam surrogate        & 4  & 2 & RE \\
RWA1  & Honeycomb heat sink          & 5  & 2 & RWA \\
\midrule
RE31  & Process design               & 3  & 3 & RE \\
RE32  & Gear train design            & 4  & 3 & RE \\
RE33  & Disc brake design            & 4  & 3 & RE \\
RE34  & Gear box design              & 5  & 3 & RE \\
RE37  & Rocket injector              & 4  & 3 & RE \\
RWA2  & Vehicle crashworthiness      & 5  & 3 & RWA \\
RWA3  & Synthesis gas production     & 3  & 3 & RWA \\
RWA4  & Wire EDM                     & 5  & 3 & RWA \\
RWA5  & Thermal storage              & 9  & 3 & RWA \\
RWA6  & Milling parameters           & 4  & 3 & RWA \\
RWA7  & Rocket injector (3-obj)      & 4  & 3 & RWA \\
\midrule
RE41  & Car side impact              & 7  & 4 & RE \\
RE42  & Supply chain design          & 6  & 4 & RE \\
RWA8  & Rocket injector (4-obj)      & 4  & 4 & RWA \\
\midrule
RWA9  & UWB antenna design           & 10 & 5 & RWA \\
RWA10 & Repellent fabric             & 3  & 7 & RWA \\
RE61  & Building design              & 3  & 6 & RE \\
RE91  & Car cab design               & 7  & 9 & RE \\
\bottomrule
\end{tabular}
\end{table}

\subsection{Per-problem tables}
\begin{table}[t]
\centering
\scriptsize
\caption{Per-problem median normalized hypervolume. $\Delta_{\text{base}}$: AOS $-$ baseline; $\Delta_{\text{rand}}$: AOS $-$ random arm.}
\label{tab:hv_per_problem}
\setlength{\tabcolsep}{3pt}
\begin{tabular}{l|rrrrr}
\toprule
\textbf{Problem} & \textbf{Baseline} & \textbf{Random arm} & \textbf{AOS} & $\Delta_{\text{base}}$ & $\Delta_{\text{rand}}$ \\
\midrule
re21 & 0.990 & \textbf{0.991} & 0.991 & 0.001 & -0.000 \\
re24 & 0.993 & \textbf{0.993} & 0.992 & -0.000 & -0.000 \\
rwa1 & 0.994 & \textbf{0.994} & 0.994 & -0.000 & -0.000 \\
\midrule
re31 & 1.000 & \textbf{1.000} & 1.000 & 0.000 & -0.000 \\
re32 & 0.999 & 0.999 & \textbf{0.999} & 0.000 & 0.000 \\
re33 & 0.997 & \textbf{0.999} & 0.998 & 0.000 & -0.001 \\
re34 & 0.977 & 0.972 & \textbf{0.977} & 0.000 & 0.005 \\
re37 & \textbf{0.935} & 0.933 & 0.935 & -0.000 & 0.002 \\
rwa2 & 0.977 & 0.972 & \textbf{0.977} & 0.000 & 0.005 \\
rwa3 & \textbf{0.883} & 0.876 & 0.881 & -0.002 & 0.006 \\
rwa4 & 0.996 & \textbf{0.997} & 0.997 & 0.000 & -0.001 \\
rwa5 & 0.980 & 0.977 & \textbf{0.981} & 0.001 & 0.004 \\
rwa6 & \textbf{0.963} & 0.959 & 0.961 & -0.001 & 0.003 \\
rwa7 & 0.961 & \textbf{0.964} & 0.964 & 0.003 & -0.000 \\
\midrule
re41 & 0.853 & 0.774 & \textbf{0.854} & 0.001 & 0.079 \\
re42 & 0.882 & \textbf{0.885} & 0.882 & 0.000 & -0.003 \\
rwa8 & 0.947 & \textbf{0.949} & 0.947 & 0.001 & -0.002 \\
\midrule
rwa9 & \textbf{0.820} & 0.758 & 0.812 & -0.008 & 0.054 \\
rwa10 & \textbf{1.097} & 1.094 & 1.081 & -0.016 & -0.013 \\
re61 & \textbf{0.998} & 0.969 & 0.989 & -0.009 & 0.020 \\
re91 & \textbf{1.538} & 1.501 & 1.490 & -0.049 & -0.011 \\
\bottomrule
\end{tabular}
\end{table}

\begin{table}[t]
\centering
\scriptsize
\caption{Per-problem median runtime (seconds). $\Delta_{\text{base}}$: AOS $-$ baseline; $\Delta_{\text{rand}}$: AOS $-$ random arm.}
\label{tab:runtime_per_problem}
\setlength{\tabcolsep}{3pt}
\begin{tabular}{l|rrrrr}
\toprule
\textbf{Problem} & \textbf{Baseline} & \textbf{Random arm} & \textbf{AOS} & $\Delta_{\text{base}}$ & $\Delta_{\text{rand}}$ \\
\midrule
cec2009\_uf1 & \textbf{6.72} & 10.14 & 10.04 & 3.32 & -0.10 \\
cec2009\_uf2 & \textbf{7.08} & 10.73 & 10.13 & 3.05 & -0.60 \\
cec2009\_uf3 & \textbf{4.70} & 11.73 & 11.78 & 7.08 & 0.05 \\
cec2009\_uf4 & \textbf{5.48} & 10.53 & 10.74 & 5.26 & 0.21 \\
cec2009\_uf5 & \textbf{4.71} & 12.05 & 10.19 & 5.49 & -1.85 \\
cec2009\_uf6 & \textbf{4.63} & 10.03 & 10.67 & 6.04 & 0.63 \\
cec2009\_uf7 & \textbf{6.35} & 12.15 & 10.20 & 3.86 & -1.95 \\
cec2009\_uf8 & \textbf{6.63} & 15.34 & 11.56 & 4.92 & -3.79 \\
cec2009\_uf9 & \textbf{6.16} & 14.45 & 9.57 & 3.41 & -4.88 \\
cec2009\_uf10 & \textbf{6.15} & 12.73 & 9.86 & 3.70 & -2.88 \\
\midrule
lsmop1 & \textbf{6.96} & 14.65 & 10.77 & 3.80 & -3.88 \\
lsmop2 & \textbf{7.62} & 14.01 & 10.06 & 2.44 & -3.95 \\
lsmop3 & \textbf{6.50} & 11.27 & 10.68 & 4.17 & -0.59 \\
lsmop4 & \textbf{10.56} & 13.38 & 11.58 & 1.02 & -1.80 \\
lsmop5 & \textbf{6.92} & 12.68 & 8.79 & 1.87 & -3.88 \\
lsmop6 & \textbf{6.56} & 9.98 & 8.05 & 1.49 & -1.93 \\
lsmop7 & \textbf{7.17} & 11.64 & 7.99 & 0.82 & -3.65 \\
lsmop8 & \textbf{7.64} & 11.69 & 8.21 & 0.57 & -3.47 \\
lsmop9 & \textbf{6.89} & 10.84 & 7.82 & 0.93 & -3.03 \\
\bottomrule
\end{tabular}
\end{table}


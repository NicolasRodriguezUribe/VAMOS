\section{Additional details}
\label{sec:appendix}

\subsection{Operator portfolio}

\paragraph{Design rationale.}
The five arms were chosen to span two axes of variation:
(i)~\emph{parameter diversity}---three SBX variants with distribution indices $\eta \in \{5, 20, 50\}$ cover an exploration--exploitation spectrum: low $\eta$ pushes offspring far from parents (broad exploration), the default $\eta{=}20$ provides balanced search, and high $\eta$ keeps offspring close to parents (fine-grained exploitation). Polynomial mutation is paired with matching $\eta$ values in arms 0--2 for coherent perturbation scales.
(ii)~\emph{structural diversity}---BLX-$\alpha$ provides a uniform interpolation geometry distinct from SBX's polynomial distribution, and DE/rand/1/bin introduces differential vectors between population members, a fundamentally different search strategy that is particularly effective on non-separable landscapes.
This layered design ensures that the portfolio contains both parameter variants of a proven operator and structurally different crossover geometries, while keeping the pool small enough ($K{=}5$) for the Thompson Sampling policy to learn meaningful reward estimates within a typical evaluation budget.

\begin{table}[t]
\centering
\scriptsize
\caption{Operator portfolio used by AOS and the random arm variant. Mutation probability is set to $1/n_{\mathrm{var}}$ for all arms.}
\label{tab:portfolio}
\setlength{\tabcolsep}{4pt}
\begin{tabular}{l|l|l|l}
\toprule
\textbf{Arm} & \textbf{Role} & \textbf{Crossover} & \textbf{Mutation} \\
\midrule
0 & Exploration      & SBX ($p{=}1.0$, $\eta{=}5$)   & PM ($\eta{=}5$)  \\
1 & Standard         & SBX ($p{=}1.0$, $\eta{=}20$)  & PM ($\eta{=}20$) \\
2 & Exploitation     & SBX ($p{=}1.0$, $\eta{=}50$)  & PM ($\eta{=}50$) \\
3 & Structural       & BLX-$\alpha$ ($p{=}0.9$, $\alpha{=}0.5$) & PM ($\eta{=}20$) \\
4 & Differential     & DE/rand/1/bin ($F{=}0.5$, $CR{=}0.9$)    & PM ($\eta{=}20$) \\
\bottomrule
\end{tabular}
\end{table}

\subsection{Benchmark problems}
\begin{table}[t]
\centering
\scriptsize
\caption{Summary of the \AOSNProblems{} benchmark problems used in the experiments.}
\label{tab:problems}
\setlength{\tabcolsep}{4pt}
\begin{tabular}{l|r|r|l}
\toprule
\textbf{Problem} & \textbf{Vars} & \textbf{Obj} & \textbf{Characteristics} \\
\midrule
UF1   & 30  & 2 & Curved PS, $f_2 = 1 - \sqrt{f_1}$ \\
UF2   & 30  & 2 & Curved PS, non-linear variable linkage \\
UF3   & 30  & 2 & Curved PS, product-based distance \\
UF4   & 30  & 2 & Concave front ($f_2 = 1 - f_1^2$) \\
UF5   & 30  & 2 & Discrete points on linear front \\
UF6   & 30  & 2 & Disconnected segments, product distance \\
UF7   & 30  & 2 & Curved PS with $f_1 = x_1^{0.2}$ \\
UF8   & 30  & 3 & Spherical front, curved PS (3-obj) \\
UF9   & 30  & 3 & Spherical front, disconnected regions (3-obj) \\
UF10  & 30  & 3 & Spherical front, cosine-based distance (3-obj) \\
\midrule
LSMOP1 & 100 & 2 & Linear front, sphere distance \\
LSMOP2 & 100 & 2 & Linear front, Griewank / Schwefel distance \\
LSMOP3 & 100 & 2 & Linear front, Rastrigin / Rosenbrock distance \\
LSMOP4 & 100 & 2 & Linear front, Ackley / Griewank distance \\
LSMOP5 & 100 & 2 & Convex front, sphere distance \\
LSMOP6 & 100 & 2 & Convex front, Rosenbrock / Schwefel distance \\
LSMOP7 & 100 & 2 & Convex front, Ackley / Rosenbrock distance \\
LSMOP8 & 100 & 2 & Convex front, Griewank / sphere distance \\
LSMOP9 & 100 & 2 & Disconnected front, sphere / Ackley distance \\
\midrule
C1-DTLZ1 & 12 & 2 & Constrained DTLZ1, linear constraint \\
C1-DTLZ3 & 12 & 2 & Constrained DTLZ3, narrow feasible region \\
C2-DTLZ2 & 12 & 2 & Constrained DTLZ2, non-convex constraint \\
\midrule
DC1-DTLZ1 & 12 & 2 & Discontinuous constraint on DTLZ1 \\
DC1-DTLZ3 & 12 & 2 & Discontinuous constraint on DTLZ3 \\
DC2-DTLZ1 & 12 & 2 & Dynamic constraint on DTLZ1 \\
DC2-DTLZ3 & 12 & 2 & Dynamic constraint on DTLZ3 \\
\midrule
MW1  & 15 & 2 & Linear constraint, convex front \\
MW2  & 15 & 2 & Non-linear constraint, convex front \\
MW3  & 15 & 2 & Disconnected feasible region \\
MW5  & 15 & 2 & Multiple constraints, narrow feasible band \\
MW6  & 15 & 2 & Large infeasible region \\
MW7  & 15 & 2 & Equality-like constraint \\
\bottomrule
\end{tabular}
\end{table}

\subsection{Per-problem tables}
\begin{table}[t]
\centering
\scriptsize
\caption{Per-problem median normalized hypervolume. $\Delta_{\text{base}}$: AOS $-$ baseline; $\Delta_{\text{rand}}$: AOS $-$ random arm.}
\label{tab:hv_per_problem}
\setlength{\tabcolsep}{3pt}
\begin{tabular}{l|rrrrr}
\toprule
\textbf{Problem} & \textbf{Baseline} & \textbf{Random arm} & \textbf{AOS} & $\Delta_{\text{base}}$ & $\Delta_{\text{rand}}$ \\
\midrule
re21 & 0.990 & \textbf{0.991} & 0.991 & 0.001 & -0.000 \\
re24 & 0.993 & \textbf{0.993} & 0.992 & -0.000 & -0.000 \\
rwa1 & 0.994 & \textbf{0.994} & 0.994 & -0.000 & -0.000 \\
\midrule
re31 & 1.000 & \textbf{1.000} & 1.000 & 0.000 & -0.000 \\
re32 & 0.999 & 0.999 & \textbf{0.999} & 0.000 & 0.000 \\
re33 & 0.997 & \textbf{0.999} & 0.998 & 0.000 & -0.001 \\
re34 & 0.977 & 0.972 & \textbf{0.977} & 0.000 & 0.005 \\
re37 & \textbf{0.935} & 0.933 & 0.935 & -0.000 & 0.002 \\
rwa2 & 0.977 & 0.972 & \textbf{0.977} & 0.000 & 0.005 \\
rwa3 & \textbf{0.883} & 0.876 & 0.881 & -0.002 & 0.006 \\
rwa4 & 0.996 & \textbf{0.997} & 0.997 & 0.000 & -0.001 \\
rwa5 & 0.980 & 0.977 & \textbf{0.981} & 0.001 & 0.004 \\
rwa6 & \textbf{0.963} & 0.959 & 0.961 & -0.001 & 0.003 \\
rwa7 & 0.961 & \textbf{0.964} & 0.964 & 0.003 & -0.000 \\
\midrule
re41 & 0.853 & 0.774 & \textbf{0.854} & 0.001 & 0.079 \\
re42 & 0.882 & \textbf{0.885} & 0.882 & 0.000 & -0.003 \\
rwa8 & 0.947 & \textbf{0.949} & 0.947 & 0.001 & -0.002 \\
\midrule
rwa9 & \textbf{0.820} & 0.758 & 0.812 & -0.008 & 0.054 \\
rwa10 & \textbf{1.097} & 1.094 & 1.081 & -0.016 & -0.013 \\
re61 & \textbf{0.998} & 0.969 & 0.989 & -0.009 & 0.020 \\
re91 & \textbf{1.538} & 1.501 & 1.490 & -0.049 & -0.011 \\
\bottomrule
\end{tabular}
\end{table}

\begin{table}[t]
\centering
\scriptsize
\caption{Per-problem median runtime (seconds). $\Delta_{\text{base}}$: AOS $-$ baseline; $\Delta_{\text{rand}}$: AOS $-$ random arm.}
\label{tab:runtime_per_problem}
\setlength{\tabcolsep}{3pt}
\begin{tabular}{l|rrrrr}
\toprule
\textbf{Problem} & \textbf{Baseline} & \textbf{Random arm} & \textbf{AOS} & $\Delta_{\text{base}}$ & $\Delta_{\text{rand}}$ \\
\midrule
cec2009\_uf1 & \textbf{6.72} & 10.14 & 10.04 & 3.32 & -0.10 \\
cec2009\_uf2 & \textbf{7.08} & 10.73 & 10.13 & 3.05 & -0.60 \\
cec2009\_uf3 & \textbf{4.70} & 11.73 & 11.78 & 7.08 & 0.05 \\
cec2009\_uf4 & \textbf{5.48} & 10.53 & 10.74 & 5.26 & 0.21 \\
cec2009\_uf5 & \textbf{4.71} & 12.05 & 10.19 & 5.49 & -1.85 \\
cec2009\_uf6 & \textbf{4.63} & 10.03 & 10.67 & 6.04 & 0.63 \\
cec2009\_uf7 & \textbf{6.35} & 12.15 & 10.20 & 3.86 & -1.95 \\
cec2009\_uf8 & \textbf{6.63} & 15.34 & 11.56 & 4.92 & -3.79 \\
cec2009\_uf9 & \textbf{6.16} & 14.45 & 9.57 & 3.41 & -4.88 \\
cec2009\_uf10 & \textbf{6.15} & 12.73 & 9.86 & 3.70 & -2.88 \\
\midrule
lsmop1 & \textbf{6.96} & 14.65 & 10.77 & 3.80 & -3.88 \\
lsmop2 & \textbf{7.62} & 14.01 & 10.06 & 2.44 & -3.95 \\
lsmop3 & \textbf{6.50} & 11.27 & 10.68 & 4.17 & -0.59 \\
lsmop4 & \textbf{10.56} & 13.38 & 11.58 & 1.02 & -1.80 \\
lsmop5 & \textbf{6.92} & 12.68 & 8.79 & 1.87 & -3.88 \\
lsmop6 & \textbf{6.56} & 9.98 & 8.05 & 1.49 & -1.93 \\
lsmop7 & \textbf{7.17} & 11.64 & 7.99 & 0.82 & -3.65 \\
lsmop8 & \textbf{7.64} & 11.69 & 8.21 & 0.57 & -3.47 \\
lsmop9 & \textbf{6.89} & 10.84 & 7.82 & 0.93 & -3.03 \\
\bottomrule
\end{tabular}
\end{table}


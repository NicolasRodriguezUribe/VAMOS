\section{Results}
\label{sec:results}

We first compare the three methods at the objective-count group level (2-obj, 3-obj, 4-obj, many-obj), then inspect per-problem deltas, convergence behavior, and operator usage.
Unless stated otherwise, we report medians over seeds.

\subsection{Solution quality}
Table~\ref{tab:hv_group} shows median normalized hypervolume grouped by number of objectives.
AOS matches the baseline overall (both achieve a median HV of \AOSHVMedianBaseline{}) and outperforms it on \AOSWins{} of \AOSNProblems{} problems, with the largest gain on \texttt{\AOSBestProblem{}} ($\Delta$=\AOSBestDelta{}) and the largest deficit on \texttt{\AOSWorstProblem{}} ($\Delta$=\AOSWorstDelta{}).
Crucially, the Wilcoxon tests (Table~\ref{tab:stat_test}) show \StatWinsBase{} significant wins and \emph{zero} significant losses against the baseline, indicating that AOS never significantly degrades performance.
Against the random arm, AOS achieves a higher median HV (\AOSHVMedianAOS{} vs.\ \AOSHVMedianRandom{}) with \StatWinsRand{} significant wins.
The comparison between random arm and baseline isolates the contribution of portfolio diversity alone; the comparison between AOS and random arm isolates the contribution of intelligent selection.

\begin{table}[t]
\centering
\caption{Mean normalized hypervolume by benchmark family (baseline vs.\ random arm vs.\ AOS, Thompson Sampling).}
\label{tab:hv_family}
\begin{tabular}{l|rrr|r}
\toprule
\textbf{Variant} & \textbf{ZDT} & \textbf{DTLZ} & \textbf{WFG} & \textbf{Average} \\
\midrule
Baseline & 0.988 & 0.793 & \textbf{0.802} & 0.861 \\
Random arm & 0.989 & 0.845 & 0.785 & 0.873 \\
AOS & \textbf{0.989} & \textbf{0.870} & 0.801 & \textbf{0.887} \\
\bottomrule
\end{tabular}
\end{table}


\begin{figure}[t]
\centering
\IfFileExists{figures/hv_delta_by_problem.pdf}{
  \includegraphics[width=\linewidth]{figures/hv_delta_by_problem.pdf}
}{
  \fbox{\parbox{0.95\linewidth}{\vspace{2mm}Missing: figures/hv\_delta\_by\_problem.pdf (run the asset script).\vspace{2mm}}}
}
\caption{Per-problem median hypervolume change (AOS $-$ Baseline), sorted by $\Delta$. Positive bars indicate problems where AOS outperforms the fixed operator.}
\label{fig:hv_delta}
\end{figure}

\begin{figure}[t]
\centering
\IfFileExists{figures/hv_delta_vs_random.pdf}{
  \includegraphics[width=\linewidth]{figures/hv_delta_vs_random.pdf}
}{
  \fbox{\parbox{0.95\linewidth}{\vspace{2mm}Missing: figures/hv\_delta\_vs\_random.pdf (run the asset script).\vspace{2mm}}}
}
\caption{Per-problem median hypervolume change (AOS $-$ Random arm), sorted by $\Delta$. Positive bars indicate problems where intelligent selection outperforms uniform random arm selection.}
\label{fig:hv_delta_rand}
\end{figure}

\subsection{Statistical significance}
To assess whether the observed differences are statistically reliable, we apply the Wilcoxon signed-rank test~\cite{wilcoxon1945} (paired, two-sided) across the \AOSNSeeds{} seeds for each problem, comparing (i)~AOS vs.\ Baseline and (ii)~AOS vs.\ Random arm on the final normalized hypervolume.
We use a significance level of $\alpha=0.05$ with Holm--Bonferroni correction for multiple comparisons across the \AOSNProblems{} problems.
Table~\ref{tab:stat_test} summarizes the win/tie/loss counts.
AOS is statistically significantly better than the baseline on \StatWinsBase{} problems, while being significantly worse on \StatLossesBase{} problems (with \StatTiesBase{} ties).
Against the random arm, AOS achieves \StatWinsRand{} significant wins and \StatLossesRand{} significant losses (\StatTiesRand{} ties), confirming that intelligent selection provides a measurable advantage over uniform random arm selection.

\begin{table}[t]
\centering
\small
\caption{Win/Tie/Loss counts (Wilcoxon signed-rank, $\alpha{=}0.05$, Holm--Bonferroni corrected). A ``win'' means AOS is significantly better; ``loss'' means significantly worse.}
\label{tab:stat_test}
\begin{tabular}{l|ccc}
\toprule
\textbf{Comparison} & \textbf{Win} & \textbf{Tie} & \textbf{Loss} \\
\midrule
AOS vs.\ Baseline   & 6 & 10 & 5 \\
AOS vs.\ Random arm & 8 & 13 & 0 \\
\bottomrule
\end{tabular}
\end{table}


\subsection{Runtime}
Table~\ref{tab:runtime_group} reports median runtime by objective-count group.
The median runtime overhead of AOS over the baseline is \AOSRuntimeOverheadPct{}\% (\AOSRuntimeMedianAOS{}s vs.\ \AOSRuntimeMedianBaseline{}s), which reflects extra policy updates, reward computation, and logging.
Interestingly, the random arm is even slower (\AOSRuntimeMedianRandom{}s) because it selects all operators with equal probability, including multi-parent crossovers (UNDX, PCX, Simplex) that are inherently more expensive; AOS learns to favor cheaper operators when they perform comparably, yielding an implicit computational benefit.
In settings where objective evaluations are expensive (e.g., simulation-based), this overhead is expected to be less pronounced in relative terms.

\begin{table}[t]
\centering
\caption{Median runtime (seconds) by benchmark family (baseline vs.\ random arm vs.\ AOS).}
\label{tab:runtime_family}
\begin{tabular}{l|rrr|r}
\toprule
\textbf{Variant} & \textbf{ZDT} & \textbf{DTLZ} & \textbf{WFG} & \textbf{Average} \\
\midrule
Baseline & \textbf{5.35} & \textbf{7.10} & \textbf{7.86} & \textbf{6.77} \\
Random arm & 8.50 & 8.74 & 9.96 & 9.07 \\
AOS & 7.89 & 8.35 & 9.42 & 8.55 \\
\bottomrule
\end{tabular}
\end{table}


\subsection{Convergence and operator usage}
Figure~\ref{fig:anytime} shows convergence in normalized hypervolume for representative problems (one per objective-count group).
Figure~\ref{fig:usage} summarizes operator selection fractions by stage for the trace-exported problems.
We observe that on some problems the policy quickly locks into a single arm, while on others it maintains diversity in operator usage, suggesting genuine landscape-driven adaptation.

\begin{figure}[t]
\centering
\IfFileExists{figures/anytime_hv_selected.pdf}{
  \includegraphics[width=\linewidth]{figures/anytime_hv_selected.pdf}
}{
  \fbox{\parbox{0.95\linewidth}{\vspace{2mm}Missing: figures/anytime\_hv\_selected.pdf (run the asset script).\vspace{2mm}}}
}
\caption{Mean$\pm$std normalized hypervolume vs.\ evaluations for selected problems (baseline vs.\ random arm vs.\ AOS).}
\label{fig:anytime}
\end{figure}

\begin{figure}[t]
\centering
\IfFileExists{figures/aos_operator_usage.pdf}{
  \includegraphics[width=\linewidth]{figures/aos_operator_usage.pdf}
}{
  \fbox{\parbox{0.95\linewidth}{\vspace{2mm}Missing: figures/aos\_operator\_usage.pdf (run the asset script).\vspace{2mm}}}
}
\caption{Mean operator selection fractions by search stage for trace-exported problems (AOS variant).}
\label{fig:usage}
\end{figure}

\subsection{Reward dynamics}
Figure~\ref{fig:reward_evolution} shows how the per-generation reward and the selected arm evolve over the course of a representative run.
Early generations exhibit high reward variance as the policy explores all arms during warmup; as estimates stabilize, the policy converges toward the best-performing arm while the exploration floor maintains occasional diversity.
This trajectory illustrates the interplay between exploitation and exploration that the AOS layer manages automatically.

\begin{figure}[t]
\centering
\IfFileExists{figures/reward_evolution.pdf}{
  \includegraphics[width=\linewidth]{figures/reward_evolution.pdf}
}{
  \fbox{\parbox{0.95\linewidth}{\vspace{2mm}Missing: figures/reward\_evolution.pdf (run the asset script).\vspace{2mm}}}
}
\caption{Per-generation reward (top) and selected arm index (bottom) for a representative problem. The warmup phase is visible in the first generations; afterward the policy converges while the exploration floor ensures occasional arm switches.}
\label{fig:reward_evolution}
\end{figure}

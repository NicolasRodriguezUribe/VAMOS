\section{Results}
\label{sec:results}

We first compare the three methods at the benchmark-family level (UF, LSMOP), then inspect convergence speed, per-problem deltas, and operator usage.
Unless stated otherwise, we report means over \AOSNSeeds{} seeds.

\subsection{Solution quality}
Table~\ref{tab:hv_family} shows mean normalized hypervolume grouped by benchmark family.
AOS achieves the highest overall mean HV (\AOSHVMeanAOS{}) compared to the baseline (\AOSHVMeanBaseline{}) and random arm (\AOSHVMeanRandom{}), outperforming the baseline on \AOSWins{} of \AOSNProblems{} problems.
The improvements are substantial:
on the UF family, AOS achieves a +7\% HV advantage over the baseline, winning on 8 of 10 problems (including UF9, where the baseline previously dominated);
on LSMOP, the advantage grows to +56\%, with AOS producing non-zero HV on problems where the baseline completely fails (e.g., LSMOP8: AOS 0.252 vs.\ baseline 0.000).
The Wilcoxon tests (Table~\ref{tab:stat_test}) confirm \StatWinsBase{} significant wins and \StatLossesBase{} significant losses against the baseline after Holm--Bonferroni correction.

\begin{table}[t]
\centering
\caption{Mean normalized hypervolume by benchmark family (baseline vs.\ random arm vs.\ AOS, Thompson Sampling).}
\label{tab:hv_family}
\begin{tabular}{l|rrr|r}
\toprule
\textbf{Variant} & \textbf{ZDT} & \textbf{DTLZ} & \textbf{WFG} & \textbf{Average} \\
\midrule
Baseline & 0.988 & 0.793 & \textbf{0.802} & 0.861 \\
Random arm & 0.989 & 0.845 & 0.785 & 0.873 \\
AOS & \textbf{0.989} & \textbf{0.870} & 0.801 & \textbf{0.887} \\
\bottomrule
\end{tabular}
\end{table}


\subsection{Convergence speed}
\label{sec:convergence}
Figure~\ref{fig:anytime_agg} shows the mean normalized HV across all \AOSNProblems{} problems at each evaluation checkpoint.
The AOS advantage grows monotonically over time: from +\AOSConvergenceAdvPct{}\% at 10{,}000 evaluations to +18\% at the full 100{,}000-evaluation budget.
This is the opposite pattern from standard benchmarks (where convergence gaps typically narrow)---here AOS keeps \emph{widening} the gap, indicating that the diverse portfolio compounds its advantage as the search progresses into harder regions of the decision space.
After generation 300 ($\approx$30K evaluations), arm elimination activates, pruning underperforming operators on problems where a single arm dominates. This stabilizes performance on tri-objective UF problems while preserving the portfolio's diversity advantage on LSMOP.

\begin{figure}[t]
\centering
\IfFileExists{figures/anytime_hv_aggregate.pdf}{
  \includegraphics[width=0.75\linewidth]{figures/anytime_hv_aggregate.pdf}
}{
  \fbox{\parbox{0.95\linewidth}{\vspace{2mm}Missing: figures/anytime\_hv\_aggregate.pdf (run the asset script).\vspace{2mm}}}
}
\caption{Mean normalized HV across all \AOSNProblems{} problems vs.\ evaluations. AOS and the random arm both substantially outperform the fixed baseline, with the advantage growing over time rather than diminishing.}
\label{fig:anytime_agg}
\end{figure}

Figure~\ref{fig:anytime} shows per-problem convergence for four representative problems where the operator-choice effect is most visible.

\begin{figure}[t]
\centering
\IfFileExists{figures/anytime_hv_selected.pdf}{
  \includegraphics[width=\linewidth]{figures/anytime_hv_selected.pdf}
}{
  \fbox{\parbox{0.95\linewidth}{\vspace{2mm}Missing: figures/anytime\_hv\_selected.pdf (run the asset script).\vspace{2mm}}}
}
\caption{Mean$\pm$std normalized HV vs.\ evaluations for selected problems. UF1 and UF7 show AOS converging rapidly to near-optimal HV, far outpacing the baseline. LSMOP1 and LSMOP4 demonstrate the dramatic advantage of a diverse portfolio in high-dimensional spaces.}
\label{fig:anytime}
\end{figure}

\subsection{Per-problem analysis}

\begin{figure}[t]
\centering
\IfFileExists{figures/hv_delta_by_problem.pdf}{
  \includegraphics[width=\linewidth]{figures/hv_delta_by_problem.pdf}
}{
  \fbox{\parbox{0.95\linewidth}{\vspace{2mm}Missing: figures/hv\_delta\_by\_problem.pdf (run the asset script).\vspace{2mm}}}
}
\caption{Per-problem mean HV change (AOS $-$ Baseline), sorted by $\Delta$. Positive bars indicate problems where AOS outperforms the fixed operator. The large positive bars on LSMOP and UF problems demonstrate AOS's strength on challenging landscapes.}
\label{fig:hv_delta}
\end{figure}

Figure~\ref{fig:hv_delta} shows per-problem HV deltas.
The pattern reveals a clear family-dependent effect:
\begin{itemize}
  \item \textbf{UF problems}: AOS achieves wins on 8 of 10 UF problems, including all 7 bi-objective problems (UF1--UF7) and the tri-objective UF9, which flipped from a substantial baseline advantage to an AOS win thanks to arm elimination. Improvements range from +0.7\% (UF2, UF4) to +17\% (UF7). The remaining two tri-objective problems (UF8, UF10) are essentially tied with the baseline (deltas $<$0.01), demonstrating that elimination successfully recovers baseline-level performance when the default operator is well-suited.
  \item \textbf{LSMOP problems}: AOS wins on 5 of 9 problems, with substantial improvements on LSMOP1 (+406\%), LSMOP4 (+44\%), and LSMOP8 (from zero to 0.252). Four LSMOP problems (LSMOP3, 6, 7, 9---those with Rastrigin, Rosenbrock, and Ackley distance functions) prove intractable for all methods at 100K evaluations.
\end{itemize}

\subsection{Statistical significance}
We apply the Wilcoxon signed-rank test~\cite{wilcoxon1945} (paired, two-sided) across \AOSNSeeds{} seeds for each problem, with Holm--Bonferroni correction for \AOSNProblems{} comparisons at $\alpha{=}0.05$.
Table~\ref{tab:stat_test} summarizes the results.
AOS achieves \StatWinsBase{} significant wins vs.\ the baseline with \emph{zero} significant losses, confirming that the gains are robust and that AOS never significantly degrades performance compared to the fixed operator.
Against the random arm, AOS achieves \StatWinsRand{} significant wins with \StatLossesRand{} significant losses; the losses reflect the cost of arm elimination, which prunes operators that the random arm continues to exploit freely.
This trade-off is deliberate: AOS sacrifices some peak diversity benefit in exchange for robustness guarantees against the baseline (Section~\ref{sec:discussion}).

\begin{table}[t]
\centering
\small
\caption{Win/Tie/Loss counts (Wilcoxon signed-rank, $\alpha{=}0.05$, Holm--Bonferroni corrected). A ``win'' means AOS is significantly better; ``loss'' means significantly worse.}
\label{tab:stat_test}
\begin{tabular}{l|ccc}
\toprule
\textbf{Comparison} & \textbf{Win} & \textbf{Tie} & \textbf{Loss} \\
\midrule
AOS vs.\ Baseline   & 6 & 10 & 5 \\
AOS vs.\ Random arm & 8 & 13 & 0 \\
\bottomrule
\end{tabular}
\end{table}


\subsection{Runtime}
Table~\ref{tab:runtime_family} reports median runtime by benchmark family.
The AOS overhead is \AOSRuntimeOverheadPct{}\% (\AOSRuntimeMedianAOS{}s vs.\ \AOSRuntimeMedianBaseline{}s), reflecting policy updates and reward computation.
The overhead is larger on these benchmarks than on standard problems because the LSMOP evaluations (100 variables) and UF evaluations (30 variables with complex functions) are computationally cheaper per evaluation, making the AOS bookkeeping a relatively larger fraction of total runtime.

\begin{table}[t]
\centering
\caption{Median runtime (seconds) by benchmark family (baseline vs.\ random arm vs.\ AOS).}
\label{tab:runtime_family}
\begin{tabular}{l|rrr|r}
\toprule
\textbf{Variant} & \textbf{ZDT} & \textbf{DTLZ} & \textbf{WFG} & \textbf{Average} \\
\midrule
Baseline & \textbf{5.35} & \textbf{7.10} & \textbf{7.86} & \textbf{6.77} \\
Random arm & 8.50 & 8.74 & 9.96 & 9.07 \\
AOS & 7.89 & 8.35 & 9.42 & 8.55 \\
\bottomrule
\end{tabular}
\end{table}


\subsection{Arm elimination and operator usage}

Figure~\ref{fig:elimination} illustrates arm elimination in action on two contrasting problems.
The top row shows operator selection fractions over generations (smoothed); the bottom row shows corresponding convergence curves.
On UF9 (a tri-objective problem where the baseline previously dominated), the elimination point at generation~300 prunes weaker arms, concentrating the budget on the most productive operators and enabling AOS to overtake the baseline.
On UF1 (a bi-objective problem where diversity helps), multiple arms remain active after elimination, preserving the portfolio's exploration benefit.

\begin{figure}[t]
\centering
\IfFileExists{figures/arm_elimination.pdf}{
  \includegraphics[width=\linewidth]{figures/arm_elimination.pdf}
}{
  \fbox{\parbox{0.95\linewidth}{\vspace{2mm}Missing: figures/arm\_elimination.pdf (run the asset script).\vspace{2mm}}}
}
\caption{Arm elimination in action. \textbf{Top}: operator selection fractions over generations (rolling average). The red dotted line marks the elimination point ($T_{\mathrm{elim}}{=}300$). On UF9, several arms are pruned; on UF1, most survive. \textbf{Bottom}: convergence curves showing that AOS overtakes the baseline on UF9 after elimination activates.}
\label{fig:elimination}
\end{figure}

Figure~\ref{fig:usage} shows operator selection fractions by search stage, and Figure~\ref{fig:reward_evolution} illustrates the reward dynamics for a representative run.

\begin{figure}[t]
\centering
\IfFileExists{figures/aos_operator_usage.pdf}{
  \includegraphics[width=\linewidth]{figures/aos_operator_usage.pdf}
}{
  \fbox{\parbox{0.95\linewidth}{\vspace{2mm}Missing: figures/aos\_operator\_usage.pdf (run the asset script).\vspace{2mm}}}
}
\caption{Mean operator selection fractions by search stage for trace-exported problems (AOS variant). The policy shifts from exploration (low-$\eta$ SBX, DE) toward exploitation (high-$\eta$ SBX) over the course of the run.}
\label{fig:usage}
\end{figure}

\begin{figure}[t]
\centering
\IfFileExists{figures/reward_evolution.pdf}{
  \includegraphics[width=\linewidth]{figures/reward_evolution.pdf}
}{
  \fbox{\parbox{0.95\linewidth}{\vspace{2mm}Missing: figures/reward\_evolution.pdf (run the asset script).\vspace{2mm}}}
}
\caption{Per-generation reward (top) and selected arm index (bottom) for a representative problem. The warm-up phase is visible in the first generations; afterward the policy converges while the exploration floor ensures occasional arm switches.}
\label{fig:reward_evolution}
\end{figure}

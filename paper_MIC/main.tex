\documentclass[runningheads]{llncs}

\usepackage{graphicx}
\usepackage{booktabs}
\usepackage{amsmath}
\usepackage{amssymb}
\usepackage[protrusion=true,expansion=false]{microtype}
\usepackage{algorithm}
\usepackage{algpseudocode}
\usepackage{hyperref}

\hypersetup{
  colorlinks=true,
  linkcolor=blue,
  citecolor=blue,
  urlcolor=blue
}

\newcommand{\VAMOS}{VAMOS}

% Auto-generated experiment summary macros (safe placeholder if not generated yet)
% Auto-generated by paper_MIC/scripts/01_make_assets.py
\newcommand{\AOSNProblems}{21}
\newcommand{\AOSNSeeds}{30}
\newcommand{\AOSWins}{9}
\newcommand{\AOSLosses}{12}
\newcommand{\AOSHVMedianBaseline}{0.934}
\newcommand{\AOSHVMedianAOS}{0.942}
\newcommand{\AOSRuntimeMedianBaseline}{1.00}
\newcommand{\AOSRuntimeMedianAOS}{1.41}
\newcommand{\AOSRuntimeOverheadPct}{41}
\newcommand{\AOSBestProblem}{dtlz6}
\newcommand{\AOSBestDelta}{0.545}
\newcommand{\AOSWorstProblem}{wfg3}
\newcommand{\AOSWorstDelta}{-0.028}


\begin{document}

\title{Adaptive Operator Selection for NSGA-II in a Vectorized Framework}
\titlerunning{Adaptive Operator Selection for NSGA-II}

\author{Nicol\'{a}s R. Uribe\inst{1} \and Alberto Herr\'{a}n\inst{1}\thanks{Corresponding author: Alberto Herr\'{a}n (alberto.herran@urjc.es)} \and Antonio J. Nebro\inst{2,3} \and J. Manuel Colmenar\inst{1}}
\authorrunning{Uribe et al.}

\institute{Dept. Computer Sciences, Universidad Rey Juan Carlos\\
C/. Tulip\'{a}n, s/n, M\'{o}stoles, 28933 (Madrid), Spain\\
\email{nicolas.rodriguez@urjc.es, alberto.herran@urjc.es, josemanuel.colmenar@urjc.es}
\and
Dept. de Lenguajes y Ciencias de la Computaci\'{o}n, ITIS Software, University of M\'{a}laga,\\
ETSI Inform\'{a}tica, Campus de Teatinos, 29071 (M\'{a}laga), Spain\\
\email{ajnebro@uma.es}
\and
ITIS Software, Ada Byron Research Building, C/. Arquitecto Francisco Pe\~nalosa, 18, University of M\'{a}laga, 29071 (M\'{a}laga), Spain\\
\email{ajnebro@uma.es}}

\maketitle

\begin{abstract}
Operator choice can strongly affect the performance of multiobjective evolutionary algorithms, yet the best operator may vary across problems and across stages of search.
We present an adaptive operator selection (AOS) layer for NSGA-II that treats each (crossover, mutation) pipeline as a bandit arm and selects one arm per generation using Thompson Sampling, updating an online policy from survival and diversity-based rewards.
We evaluate on \AOSNProblems{} standard benchmarks from the ZDT, DTLZ, and WFG families (2--3 objectives, 7--30 variables).
A three-way comparison (fixed operator vs.\ random arm vs.\ AOS) with \AOSNSeeds{} seeds reveals two main findings: (i)~AOS converges \AOSConvergenceAdvPct{}\% faster than the fixed baseline at 20\% of the evaluation budget, and (ii)~AOS achieves a higher mean normalized HV (\AOSHVMeanAOS{} vs.\ \AOSHVMeanBaseline{}) with \AOSWins{}/\AOSNProblems{} problem wins at the final budget.
On the minority of problems where the default operator is already near-optimal (concave WFG fronts), AOS incurs a small exploration cost that is consistently smaller than that of random arm selection.
AOS incurs a median runtime overhead of \AOSRuntimeOverheadPct{}\%.
\keywords{Adaptive operator selection \and Multi-armed bandits \and NSGA-II \and Multiobjective optimization \and Thompson Sampling}
\end{abstract}

\section{Introduction}
\label{sec:intro}

Multiobjective evolutionary algorithms (MOEAs) are a default choice for black-box optimization problems with conflicting objectives, including engineering design and automated decision support.
Most MOEAs rely on a small number of variation operators (crossover and mutation) to balance exploration and exploitation.
However, operator choice is a major source of performance variability: operators that work well on one problem can stall on another, and the most effective operator can change as the population approaches the Pareto front.
This motivates \emph{adaptive operator selection} (AOS): selecting operators online based on feedback from the ongoing search, rather than relying on a single hand-tuned pipeline.

Most AOS mechanisms can be seen as two coupled components: (i) credit assignment (how to score an operator based on recent offspring) and (ii) an adaptation rule (how to turn scores into future operator-selection decisions).
Following the bandit view of AOS~\cite{dacosta2008aosdmab}, we model each candidate variation pipeline as an arm in a multi-armed bandit, and update an online policy from per-generation rewards.
Compared to offline tuning, this approach aims to improve robustness across problems and reduce the need for manual operator configuration.

This paper focuses on NSGA-II~\cite{deb2002nsgaii}, a widely used MOEA whose performance depends strongly on variation settings.
We implement a generation-level AOS layer for NSGA-II in a vectorized optimization framework and evaluate it on standard benchmark suites.
Our evaluation highlights both the potential benefits of online adaptation (substantial gains on some problems) and the practical costs (runtime overhead from bookkeeping and additional computations).

\paragraph{Contributions.}
This paper makes four contributions:
\begin{itemize}
  \item A simple, reproducible AOS interface for NSGA-II that selects exactly one variation pipeline per generation.
  \item A reward design aligned with NSGA-II survivor selection, combining offspring survival and non-dominated insertion rates, with an optional bounded hypervolume-improvement proxy.
  \item An analysis workflow that logs per-generation operator choices and rewards, enabling inspection of \emph{when} and \emph{how} the policy switches operators during search.
  \item An empirical evaluation on classic MOEA benchmark suites~\cite{zitzler2000zdt,deb2002dtlz,huband2006wfg} (ZDT/\allowbreak DTLZ/\allowbreak WFG) showing that AOS can improve median normalized hypervolume on difficult instances, while incurring measurable runtime overhead.
\end{itemize}

\paragraph{Organization.}
Section~\ref{sec:background} reviews NSGA-II and the bandit perspective on AOS.
Section~\ref{sec:method} details the portfolio, reward signals, and policies.
Section~\ref{sec:implementation} summarizes implementation and reproducibility details.
Sections~\ref{sec:experiments}--\ref{sec:results} describe the experimental protocol and results, followed by discussion and threats to validity.

\section{Background and related work}
\label{sec:background}

\subsection{NSGA-II}
NSGA-II~\cite{deb2002nsgaii} maintains a population and iteratively generates offspring via variation operators.
Survivor selection is based on non-dominated sorting and crowding distance, promoting both convergence and diversity.
Although the selection mechanism is parameter-light, practical performance depends on the chosen crossover/mutation pipeline and its hyperparameters, which are often selected by convention or tuned per problem.

\subsection{Adaptive operator selection as bandits}
In the multi-armed bandit (MAB) setting, a learner repeatedly selects an arm and observes a reward, aiming to maximize cumulative reward by balancing exploration and exploitation.
Common policies include $\varepsilon$-greedy, upper-confidence bounds (UCB)~\cite{auer2002ucb1}, adversarial methods such as EXP3~\cite{auer2002exp3}, and Bayesian approaches such as Thompson sampling~\cite{thompson1933ts}.
Bandit-based AOS~\cite{dacosta2008aosdmab} maps each operator (or operator pipeline) to an arm and uses online rewards to guide selection.

\subsection{Credit assignment and non-stationarity}
In evolutionary search, operator utility is inherently non-stationary: exploration-heavy operators can be useful early, while exploitation and local refinement become important as the population converges.
This raises two practical design questions.
First, how should rewards be defined so they are informative across problems and across stages?
Second, which policies handle non-stationarity without over-reacting to noise?
Sliding-window variants of classical bandit policies (e.g., UCB) and Bayesian policies with limited-memory reward histories are common approaches.

\section{Method}
\label{sec:method}

\subsection{Bandit formulation}
Let $\mathcal{A} = \{1,\dots,K\}$ be a portfolio of $K$ candidate variation pipelines (arms).
At each generation $t$, the algorithm selects one arm $a_t \in \mathcal{A}$, generates all offspring for that generation using that arm, observes a scalar reward $r_t \in [0,1]$, and updates its policy.
The objective is to maximize cumulative reward, which serves as a proxy for improved search progress.

\subsection{Operator portfolios}
We define a small \emph{portfolio} of candidate variation pipelines, where each arm specifies one crossover operator and one mutation operator with fixed parameters.
At generation $t$, NSGA-II selects exactly one arm and uses it to generate all offspring for that generation.
This generation-level decision reduces reward noise (all offspring share the same operator) and keeps the online learning interface simple.
It also mirrors how practitioners often tune operators: choosing a pipeline and running it for a while, rather than switching per mating event.

\subsection{Reward signals}
After survivor selection, we compute a reward that summarizes how useful the selected operator was for that generation.
Let $n_{\mathrm{off}}$ be the offspring count, $n_{\mathrm{surv}}$ the number of offspring that survive into the next population, and $n_{\mathrm{nd}}$ the number of surviving offspring that are non-dominated in the next population.
We define:
\begin{align}
  r_{\mathrm{surv}} &= \frac{n_{\mathrm{surv}}}{n_{\mathrm{off}}}, &
  r_{\mathrm{nd}} &= \frac{n_{\mathrm{nd}}}{n_{\mathrm{off}}}.
\end{align}
Both rates are naturally bounded in $[0,1]$ and are available for any problem without requiring additional reference information.

\paragraph{Optional hypervolume proxy.}
When reference information is available, we include a bounded proxy for hypervolume improvement, $r_{\mathrm{hv}} \in [0,1]$.
Let $\mathrm{HV}(F)$ be the normalized hypervolume of the current population front $F$, computed with a fixed reference point.
We compute a relative improvement ratio
\begin{align}
  \rho_t = \frac{\mathrm{HV}(F_t) - \mathrm{HV}(F_{t-1})}{\max(|\mathrm{HV}(F_{t-1})|,\epsilon)},
\end{align}
and squash it to $[0,1]$ via $r_{\mathrm{hv}} = 0.5 + 0.5\tanh(\rho_t)$.
This keeps the reward scale consistent across problems and avoids unbounded updates on early generations where absolute HV values can be small.

The aggregate reward can be configured as a single component or as a convex combination:
\begin{align}
  r = w_{\mathrm{surv}} r_{\mathrm{surv}} + w_{\mathrm{nd}} r_{\mathrm{nd}} + w_{\mathrm{hv}} r_{\mathrm{hv}}.
\end{align}
In our implementation the weights are normalized to sum to one; if all are zero, we default to a balanced survival/ND reward.

\subsection{Policies and practical details}
We support several bandit policies (e.g., $\varepsilon$-greedy, UCB, EXP3, Thompson sampling) and two stabilization mechanisms:
(i) a \emph{warmup} minimum-usage rule that forces each arm to be tried at least a given number of times, and (ii) an \emph{exploration floor} that mixes in a uniform arm draw with fixed probability.
All random choices can be driven by an explicit policy RNG seed to support reproducibility.

\paragraph{Credit timing.}
Rewards are computed once per generation, after survivor selection has produced the next population.
The policy is updated once per generation using the scalar reward, which matches the generation-level selection granularity.

\begin{algorithm}[t]
\caption{NSGA-II with generation-level AOS (sketch).}
\label{alg:nsgaii_aos}
\begin{algorithmic}[1]
\For{$t = 1,2,\dots$}
  \State Select arm $a_t$ using bandit policy (with optional warmup/floor).
  \State Generate offspring using the variation pipeline of $a_t$.
  \State Apply NSGA-II survivor selection to form next population.
  \State Compute reward $r_t$ from survival/diversity signals (and optional HV proxy).
  \State Update bandit policy with $(a_t, r_t)$.
\EndFor
\end{algorithmic}
\end{algorithm}

\section{Implementation and reproducibility}
\label{sec:implementation}

\subsection{Integration into NSGA-II}
The five-stage pipeline described in Section~\ref{sec:aos_overview} is implemented as a lightweight controller that wraps the standard NSGA-II loop.
At the beginning of each generation, the controller queries the bandit policy (Stage~2) and passes the selected arm index to the variation module.
During the generation, it records the number of offspring created by the selected arm.
After survivor selection, it computes reward components ($r_{\mathrm{surv}}$, $r_{\mathrm{nd}}$, and optionally $r_{\mathrm{hv}}$), updates the policy (Stage~5), and emits a trace row for logging.
The controller adds no additional objective evaluations; its overhead comes solely from bookkeeping and the policy update.

\subsection{Determinism}
To support reproducibility, we separate (a) the global run seed that drives variation and selection randomness from (b) an optional policy RNG seed that drives the stochastic policy components (e.g., $\varepsilon$-greedy exploration).
With fixed seeds, the operator-selection sequence and rewards are deterministic for a given configuration.

\subsection{Logging}
We log two types of artifacts for post-hoc analysis:
(i) a per-generation trace (generation index, selected operator, reward breakdown, and batch size), and
(ii) a summary table (number of pulls and mean reward per arm at the end of the run).
These logs enable inspection of when the policy switches operators, whether it locks into a single arm, and how reward signals evolve over time.

\subsection{Computational efficiency}
All experiments are executed within VAMOS, a vectorized Python framework for multiobjective optimization that leverages Numba JIT compilation to achieve near-native performance.
We compare its runtime against pymoo~\cite{pymoo} and jMetalPy~\cite{jmetalpy}, the two most widely adopted Python MOEA frameworks in recent literature.
DEAP~\cite{deap} and Platypus are deliberately excluded from this comparison: DEAP is a general-purpose evolutionary computation toolkit whose NSGA-II implementation prioritizes pedagogical clarity over computational efficiency, while Platypus has seen little active maintenance and has a substantially smaller adoption in the MOEA community.
Both frameworks ran more than one order of magnitude slower than pymoo on our benchmarks in preliminary tests, making them unsuitable comparison points for a performance evaluation; retaining them would also have multiplied the total experiment time by a factor incompatible with the publication timeline.
The comparison is therefore restricted to pymoo and jMetalPy, which represent the current state of the art in optimized Python MOEA implementations.


\section{Experimental setup}
\label{sec:experiments}

\subsection{Benchmarks and budget}
We evaluate on 21 continuous multiobjective benchmarks drawn from three established families:
\begin{itemize}
  \item \textbf{ZDT}~\cite{zitzler2000zdt}: 5 bi-objective problems (ZDT1--4, ZDT6) with 10--30 decision variables.
  \item \textbf{DTLZ}~\cite{deb2002dtlz}: 7 tri-objective problems (DTLZ1--7) with 7--22 decision variables and diverse landscape features (multi-modal, degenerate, disconnected fronts).
  \item \textbf{WFG}~\cite{huband2006wfg}: 9 tri-objective problems (WFG1--9) with 24 decision variables and systematic variation in separability, modality, and front geometry.
\end{itemize}
Together these 21 problems cover a wide range of landscape characteristics---unimodal, multi-modal, deceptive, separable, non-separable, convex, concave, mixed, disconnected, and degenerate fronts---making them a standard testbed for evaluating algorithmic mechanisms in the multiobjective community.
Each run is terminated after a fixed evaluation budget of 50{,}000 function evaluations, using \AOSNSeeds{} independent seeds per problem.

\subsection{Methods compared}
We compare three methods, all based on NSGA-II with the same 5-arm operator portfolio (Appendix, Table~\ref{tab:portfolio}):

\begin{enumerate}
  \item \textbf{Baseline}: Fixed SBX + polynomial mutation (SBX crossover probability 1.0, $\eta_c=20$; polynomial mutation probability $1/n_{\mathrm{var}}$, $\eta_m=20$).
        This is the most common default operator pipeline in the MOEA literature.
  \item \textbf{Random arm}: At each generation, one of the five operator arms is selected uniformly at random.
        This isolates the effect of portfolio \emph{diversity} (having multiple operators) from \emph{intelligent selection}.
  \item \textbf{AOS}: The same portfolio, but one arm is selected per generation using Thompson Sampling with a sliding window of $W{=}50$ generations, warm-up of $m_{\min}{=}5$ pulls per arm, and an exploration floor of $p_{\mathrm{floor}}{=}0.05$.
        Rewards use a weighted combination of survival rate, non-dominated insertion rate, and a bounded hypervolume proxy with weights $(0.4, 0.4, 0.2)$.
\end{enumerate}
Comparing (1) vs.\ (2) reveals whether operator diversity alone helps; comparing (2) vs.\ (3) reveals whether the bandit policy adds value beyond random selection.

\subsection{Operator portfolio}
The five arms were designed along two axes of diversity:
(i)~\emph{parameter diversity}---three SBX variants with distribution indices $\eta \in \{5, 20, 50\}$ span an exploration--exploitation continuum, from wide offspring spread (low $\eta$) to children near parents (high $\eta$); and
(ii)~\emph{structural diversity}---BLX-$\alpha$ provides a different interpolation geometry, while DE/rand/1/bin introduces a fundamentally different search strategy using differential vectors between population members.
All arms use polynomial mutation, keeping the portfolio small ($K{=}5$) and differences attributable solely to crossover strategy.
This design is intentionally composed of variations and complements of a strong baseline (SBX+PM), reflecting the practical insight that large performance gains are more reliably achieved through calibrated parameter variation than through mixing in highly disruptive operators.

\subsection{Metrics}
We report normalized hypervolume (higher is better) and wall-clock runtime (lower is better).
Hypervolume is computed using approximate reference fronts generated by running multiple seeds of baseline NSGA-II with an extended budget (200{,}000 evaluations, 50 seeds) and merging all non-dominated solutions per problem.
The reference point is the per-objective maximum of the reference front plus a small $\epsilon$ margin.
Normalized hypervolume divides the raw HV of a run by the reference front's HV; values near~1 indicate performance comparable to the reference front (values slightly above~1 can occur when a run discovers solutions beyond the reference approximation).
We additionally report \emph{anytime convergence}: mean normalized HV at evaluation checkpoints (5\,000, 10\,000, 20\,000, 50\,000), which captures convergence \emph{speed} in addition to final quality.

\begin{table}[t]
\centering
\small
\caption{Key experimental settings.}
\label{tab:settings}
\begin{tabular}{l|l}
\toprule
Setting & Value \\
\midrule
Algorithm & NSGA-II \\
Population / offspring & 100 / 100 \\
Budget & 50{,}000 evaluations \\
Seeds & \AOSNSeeds{} per problem \\
Problems & \AOSNProblems{} (ZDT + DTLZ + WFG, continuous) \\
Baseline operators & SBX($p{=}1.0$, $\eta_c{=}20$) + PM($p{=}1/n_{\mathrm{var}}$, $\eta_m{=}20$) \\
Random arm & Uniform selection from 5-arm pool \\
AOS policy & Thompson Sampling ($W{=}50$, $m_{\min}{=}5$, $p_{\mathrm{floor}}{=}0.05$) \\
AOS reward & $0.4\,r_{\mathrm{surv}} + 0.4\,r_{\mathrm{nd}} + 0.2\,r_{\mathrm{hv}}$ \\
Portfolio & 5 arms (Table~\ref{tab:portfolio}) \\
\bottomrule
\end{tabular}
\end{table}

\section{Results}
\label{sec:results}

We first compare the three methods at the benchmark-family level (ZDT, DTLZ, WFG), then inspect convergence speed, per-problem deltas, and operator usage.
Unless stated otherwise, we report means over \AOSNSeeds{} seeds.

\subsection{Solution quality}
Table~\ref{tab:hv_family} shows mean normalized hypervolume grouped by benchmark family.
AOS achieves the highest overall mean HV (\AOSHVMeanAOS{}) compared to the baseline (\AOSHVMeanBaseline{}) and random arm (\AOSHVMeanRandom{}), outperforming the baseline on \AOSWins{} of \AOSNProblems{} problems.
The largest gain is on \texttt{\AOSBestProblem{}} ($\Delta{=}\AOSBestDelta{}$), where the baseline's fixed SBX+PM pipeline fails to converge; AOS's adaptive arm selection discovers effective operators that reach the true front.
The Wilcoxon tests (Table~\ref{tab:stat_test}) confirm \StatWinsBase{} significant wins and \StatLossesBase{} significant losses against the baseline after Holm--Bonferroni correction.
Against the random arm, AOS achieves \StatWinsRand{} significant wins with \StatLossesRand{} significant losses, confirming that intelligent selection adds measurable value beyond random operator switching.

\begin{table}[t]
\centering
\caption{Mean normalized hypervolume by benchmark family (baseline vs.\ random arm vs.\ AOS, Thompson Sampling).}
\label{tab:hv_family}
\begin{tabular}{l|rrr|r}
\toprule
\textbf{Variant} & \textbf{ZDT} & \textbf{DTLZ} & \textbf{WFG} & \textbf{Average} \\
\midrule
Baseline & 0.988 & 0.793 & \textbf{0.802} & 0.861 \\
Random arm & 0.989 & 0.845 & 0.785 & 0.873 \\
AOS & \textbf{0.989} & \textbf{0.870} & 0.801 & \textbf{0.887} \\
\bottomrule
\end{tabular}
\end{table}


\subsection{Convergence speed}
\label{sec:convergence}
The most striking result is not the final HV but the convergence \emph{speed}.
Figure~\ref{fig:anytime_agg} shows the mean normalized HV across all 21 problems at each evaluation checkpoint.
At 10{,}000 evaluations (20\% of budget), AOS achieves a mean HV of \AOSHVAOSTenK{} vs.\ the baseline's \AOSHVBaselineTenK{}---a \AOSConvergenceAdvPct{}\% advantage.
This early-stage dominance stems from AOS's ability to deploy exploratory operators (low-$\eta$ SBX, DE) when the population is spread across the search space, then shift toward exploitation operators (high-$\eta$ SBX) as the population converges.
The gap narrows toward the full budget as all methods approach the reference front, but AOS maintains its advantage throughout.

\begin{figure}[t]
\centering
\IfFileExists{figures/anytime_hv_aggregate.pdf}{
  \includegraphics[width=0.75\linewidth]{figures/anytime_hv_aggregate.pdf}
}{
  \fbox{\parbox{0.95\linewidth}{\vspace{2mm}Missing: figures/anytime\_hv\_aggregate.pdf (run the asset script).\vspace{2mm}}}
}
\caption{Mean normalized HV across all 21 problems vs.\ evaluations. AOS converges significantly faster than both the baseline and random arm selection, with the largest advantage at early checkpoints.}
\label{fig:anytime_agg}
\end{figure}

Figure~\ref{fig:anytime} shows per-problem convergence for four representative problems where AOS's convergence advantage is most visible.

\begin{figure}[t]
\centering
\IfFileExists{figures/anytime_hv_selected.pdf}{
  \includegraphics[width=\linewidth]{figures/anytime_hv_selected.pdf}
}{
  \fbox{\parbox{0.95\linewidth}{\vspace{2mm}Missing: figures/anytime\_hv\_selected.pdf (run the asset script).\vspace{2mm}}}
}
\caption{Mean$\pm$std normalized hypervolume vs.\ evaluations for selected problems. AOS converges substantially faster on ZDT2 and DTLZ7, and maintains or extends the lead through the full budget on WFG1 and WFG2.}
\label{fig:anytime}
\end{figure}

\subsection{Per-problem analysis}

\begin{figure}[t]
\centering
\IfFileExists{figures/hv_delta_by_problem.pdf}{
  \includegraphics[width=\linewidth]{figures/hv_delta_by_problem.pdf}
}{
  \fbox{\parbox{0.95\linewidth}{\vspace{2mm}Missing: figures/hv\_delta\_by\_problem.pdf (run the asset script).\vspace{2mm}}}
}
\caption{Per-problem mean HV change (AOS $-$ Baseline), sorted by $\Delta$. Positive bars indicate problems where AOS outperforms the fixed operator.}
\label{fig:hv_delta}
\end{figure}

Figure~\ref{fig:hv_delta} shows per-problem HV deltas.
The pattern reveals a family-dependent effect:
AOS achieves its strongest gains on DTLZ and ZDT problems, where landscape features (multi-modality, degenerate fronts, disconnected regions) reward operator adaptation.
On WFG4--9, which feature concave fronts where SBX with $\eta{=}20$ is already near-optimal, AOS shows a small deficit attributable to the exploration cost of trying alternative arms.
Crucially, on every WFG problem where AOS loses, the random arm loses even more (Section~\ref{sec:discussion}), confirming that the deficit is not a failure of the policy but an inherent cost of maintaining a diverse portfolio.

\subsection{Statistical significance}
We apply the Wilcoxon signed-rank test~\cite{wilcoxon1945} (paired, two-sided) across \AOSNSeeds{} seeds for each problem, with Holm--Bonferroni correction for \AOSNProblems{} comparisons at $\alpha{=}0.05$.
Table~\ref{tab:stat_test} summarizes the results.
AOS achieves \StatWinsBase{} significant wins vs.\ the baseline with \StatLossesBase{} losses, and \StatWinsRand{} significant wins vs.\ the random arm.

\begin{table}[t]
\centering
\small
\caption{Win/Tie/Loss counts (Wilcoxon signed-rank, $\alpha{=}0.05$, Holm--Bonferroni corrected). A ``win'' means AOS is significantly better; ``loss'' means significantly worse.}
\label{tab:stat_test}
\begin{tabular}{l|ccc}
\toprule
\textbf{Comparison} & \textbf{Win} & \textbf{Tie} & \textbf{Loss} \\
\midrule
AOS vs.\ Baseline   & 6 & 10 & 5 \\
AOS vs.\ Random arm & 8 & 13 & 0 \\
\bottomrule
\end{tabular}
\end{table}


\subsection{Runtime}
Table~\ref{tab:runtime_family} reports median runtime by benchmark family.
The AOS overhead is \AOSRuntimeOverheadPct{}\% (\AOSRuntimeMedianAOS{}s vs.\ \AOSRuntimeMedianBaseline{}s), reflecting policy updates and reward computation.
The random arm is slower (\AOSRuntimeMedianRandom{}s) because it selects all operators with equal probability, including the 3-parent DE crossover; AOS learns to favor efficient operators, yielding an implicit computational benefit.

\begin{table}[t]
\centering
\caption{Median runtime (seconds) by benchmark family (baseline vs.\ random arm vs.\ AOS).}
\label{tab:runtime_family}
\begin{tabular}{l|rrr|r}
\toprule
\textbf{Variant} & \textbf{ZDT} & \textbf{DTLZ} & \textbf{WFG} & \textbf{Average} \\
\midrule
Baseline & \textbf{5.35} & \textbf{7.10} & \textbf{7.86} & \textbf{6.77} \\
Random arm & 8.50 & 8.74 & 9.96 & 9.07 \\
AOS & 7.89 & 8.35 & 9.42 & 8.55 \\
\bottomrule
\end{tabular}
\end{table}


\subsection{Operator usage}
Figure~\ref{fig:usage} shows operator selection fractions by search stage for the trace-exported problems.
Thompson Sampling learns to shift usage from exploratory arms toward exploitation arms as the search progresses, confirming genuine landscape-driven adaptation.
Figure~\ref{fig:reward_evolution} illustrates the reward dynamics for a representative run.

\begin{figure}[t]
\centering
\IfFileExists{figures/aos_operator_usage.pdf}{
  \includegraphics[width=\linewidth]{figures/aos_operator_usage.pdf}
}{
  \fbox{\parbox{0.95\linewidth}{\vspace{2mm}Missing: figures/aos\_operator\_usage.pdf (run the asset script).\vspace{2mm}}}
}
\caption{Mean operator selection fractions by search stage for trace-exported problems (AOS variant). The policy shifts from exploration (low-$\eta$ SBX, DE) toward exploitation (high-$\eta$ SBX) over the course of the run.}
\label{fig:usage}
\end{figure}

\begin{figure}[t]
\centering
\IfFileExists{figures/reward_evolution.pdf}{
  \includegraphics[width=\linewidth]{figures/reward_evolution.pdf}
}{
  \fbox{\parbox{0.95\linewidth}{\vspace{2mm}Missing: figures/reward\_evolution.pdf (run the asset script).\vspace{2mm}}}
}
\caption{Per-generation reward (top) and selected arm index (bottom) for a representative problem. The warm-up phase is visible in the first generations; afterward the policy converges while the exploration floor ensures occasional arm switches.}
\label{fig:reward_evolution}
\end{figure}

\section{Discussion}
\label{sec:discussion}

The results show that AOS with Thompson Sampling provides two distinct advantages: (i)~significantly faster convergence across the full benchmark suite, and (ii)~higher or comparable final quality on the majority of problems.
We now analyze the mechanisms behind both the wins and the losses.

\paragraph{Convergence speed is the primary benefit.}
The aggregate anytime curves (Figure~\ref{fig:anytime_agg}) reveal that AOS's strongest advantage is in the first 10\% of the evaluation budget.
At 10{,}000 evaluations the mean HV advantage over the baseline is \AOSConvergenceAdvPct{}\%.
This arises because Thompson Sampling actively explores the operator space during early generations---deploying low-$\eta$ SBX and DE to spread offspring broadly---then progressively shifts toward exploitation arms as reward estimates stabilize.
The baseline, locked into a single moderate operator ($\eta{=}20$), converges more slowly when the initial population is far from the Pareto front.
Crucially, AOS maintains a positive advantage throughout the full 100{,}000-evaluation budget (+2.3\% at termination), confirming that the early convergence gains are not simply ``borrowed'' from later stages.
In practical applications where evaluation budgets are limited (e.g., simulation-based engineering design), this faster convergence translates directly into better solutions for a given computational cost.

\paragraph{The exploration--exploitation trade-off on WFG.}
AOS loses to the baseline on 4 of 9 WFG problems (WFG4, 5, 7, 8).
A deeper analysis reveals that these losses are inherent to the portfolio, not the policy:
on every WFG problem where AOS loses, the random arm loses \emph{even more} (e.g., WFG8: AOS $-5.0\%$ vs.\ random $-7.4\%$; WFG7: AOS $-2.5\%$ vs.\ random $-3.8\%$).
This means that \emph{any} departure from pure SBX+PM ($\eta{=}20$) hurts on these specific landscapes---concave fronts where the default operator happens to be near-optimal.
AOS mitigates much of the portfolio cost through intelligent selection (the trace data shows $\sim$80\% of pulls going to SBX arms on WFG problems), but even 10--20\% exploration time on weaker arms creates a small deficit.
This is a fundamental exploration cost: the portfolio pays a small tax ($\sim$2--5\%) on easy landscapes in exchange for large gains ($+9$--$64\%$) on hard ones.

\paragraph{Where AOS helps most.}
The largest improvements occur on problems where the default SBX+PM pipeline stalls or converges slowly:
\begin{itemize}
  \item \textbf{DTLZ6} ($+64\%$): The baseline struggles with the degenerate, biased distance function; AOS's low-$\eta$ and DE arms explore the decision space more broadly, reaching the true Pareto front.
  \item \textbf{WFG2} ($+9\%$): The disconnected, non-separable front benefits from structural diversity in the crossover operators.
  \item \textbf{DTLZ3} ($+1.8\%$) \textbf{and DTLZ7} ($+1.5\%$): Multi-modal landscapes and disconnected fronts reward operators that can maintain population spread.
  \item \textbf{WFG1} ($+1.3\%$): The mixed convex/flat front benefits from AOS selecting appropriate operators for different convergence phases.
\end{itemize}

\paragraph{The value of intelligent selection.}
The three-way design separates portfolio diversity from intelligent selection.
On the aggregate, the random arm (uniform operator switching) outperforms the baseline (mean HV \AOSHVMeanRandom{} vs.\ \AOSHVMeanBaseline{}), showing that diversity alone has value.
AOS then improves further over the random arm (\AOSHVMeanAOS{}), demonstrating that Thompson Sampling adds measurable benefit by concentrating pulls on the most rewarding operators for each problem.
The statistical tests confirm this: AOS achieves \StatWinsRand{} significant wins against the random arm.

\paragraph{Practical implications.}
Our results suggest a practical guideline: an AOS layer based on Thompson Sampling with a parameter-diverse SBX portfolio is a safe default for NSGA-II.
On \AOSWins{} of \AOSNProblems{} standard benchmarks it improves over the fixed default, and on the remaining \AOSLosses{} the degradation is small and always less than what random switching would produce.
The convergence speed advantage makes AOS particularly attractive in budget-constrained settings.
For practitioners who know their landscape favors SBX with a specific $\eta$ (e.g., WFG-style concave fronts), the fixed operator remains a valid choice; otherwise, AOS provides automatic adaptation with a favorable risk--reward profile.

\section{Threats to validity}
\label{sec:threats}

\paragraph{Portfolio dependence.}
Our conclusions depend on the chosen operator portfolio.
A different set of arms (or different parameterizations) could shift both the magnitude and direction of the observed effects.

\paragraph{Reward design.}
We use a simple reward based on survival and non-dominated insertion, with an optional hypervolume proxy.
More expressive credit assignment schemes (e.g., multi-step returns, per-offspring credit, or diversity-aware shaping) may produce different learning dynamics.

\paragraph{Benchmark scope.}
We evaluate on continuous engineering surrogates (RE and RWA suites) with 2--9 objectives and 3--10 decision variables.
Results may not transfer directly to constrained problems, discrete encodings, higher-dimensional search spaces, or real-world objectives with expensive evaluations where overhead amortization differs.

\paragraph{Runtime measurement.}
Runtime overhead is measured in a vectorized engine, making controller overhead more visible.
In other settings, the relative overhead of AOS may be smaller.

\section{Conclusion}
\label{sec:conclusion}

We presented a bandit-based adaptive operator selection layer for NSGA-II that selects one variation pipeline per generation and updates an online policy from survival and diversity-oriented rewards.
Across standard benchmark suites, AOS improves median normalized hypervolume on a subset of problems and substantially improves performance on DTLZ6, while incurring a median runtime overhead of \AOSRuntimeOverheadPct{}\%.

Future work includes portfolio construction rules, more informative reward signals, non-stationary policies with explicit change detection, and cost-aware implementations that reduce controller overhead (e.g., by reusing selection statistics already computed by NSGA-II).



\appendix
\renewcommand{\theHsection}{app.\Alph{section}}
\section{Additional details}
\label{sec:appendix}

\subsection{Operator portfolio}

\paragraph{Design rationale.}
The five arms were chosen to span two axes of variation:
(i)~\emph{parameter diversity}---three SBX variants with distribution indices $\eta \in \{5, 20, 50\}$ cover an exploration--exploitation spectrum: low $\eta$ pushes offspring far from parents (broad exploration), the default $\eta{=}20$ provides balanced search, and high $\eta$ keeps offspring close to parents (fine-grained exploitation). Polynomial mutation is paired with matching $\eta$ values in arms 0--2 for coherent perturbation scales.
(ii)~\emph{structural diversity}---BLX-$\alpha$ provides a uniform interpolation geometry distinct from SBX's polynomial distribution, and DE/rand/1/bin introduces differential vectors between population members, a fundamentally different search strategy that is particularly effective on non-separable landscapes.
This layered design ensures that the portfolio contains both parameter variants of a proven operator and structurally different crossover geometries, while keeping the pool small enough ($K{=}5$) for the Thompson Sampling policy to learn meaningful reward estimates within a typical evaluation budget.

\begin{table}[t]
\centering
\scriptsize
\caption{Operator portfolio used by AOS and the random arm variant. Mutation probability is set to $1/n_{\mathrm{var}}$ for all arms.}
\label{tab:portfolio}
\setlength{\tabcolsep}{4pt}
\begin{tabular}{l|l|l|l}
\toprule
\textbf{Arm} & \textbf{Role} & \textbf{Crossover} & \textbf{Mutation} \\
\midrule
0 & Exploration      & SBX ($p{=}1.0$, $\eta{=}5$)   & PM ($\eta{=}5$)  \\
1 & Standard         & SBX ($p{=}1.0$, $\eta{=}20$)  & PM ($\eta{=}20$) \\
2 & Exploitation     & SBX ($p{=}1.0$, $\eta{=}50$)  & PM ($\eta{=}50$) \\
3 & Structural       & BLX-$\alpha$ ($p{=}0.9$, $\alpha{=}0.5$) & PM ($\eta{=}20$) \\
4 & Differential     & DE/rand/1/bin ($F{=}0.5$, $CR{=}0.9$)    & PM ($\eta{=}20$) \\
\bottomrule
\end{tabular}
\end{table}

\subsection{Benchmark problems}
\begin{table}[t]
\centering
\scriptsize
\caption{Summary of the 21 benchmark problems used in the experiments.}
\label{tab:problems}
\setlength{\tabcolsep}{4pt}
\begin{tabular}{l|r|r|l}
\toprule
\textbf{Problem} & \textbf{Vars} & \textbf{Obj} & \textbf{Characteristics} \\
\midrule
ZDT1  & 30 & 2 & Convex front \\
ZDT2  & 30 & 2 & Non-convex front \\
ZDT3  & 30 & 2 & Disconnected front \\
ZDT4  & 10 & 2 & Multi-modal ($21^9$ local fronts) \\
ZDT6  & 10 & 2 & Non-uniform density, biased \\
\midrule
DTLZ1 & 7  & 3 & Linear front, multi-modal ($11^5$ local fronts) \\
DTLZ2 & 12 & 3 & Concave front \\
DTLZ3 & 12 & 3 & Concave front, highly multi-modal \\
DTLZ4 & 12 & 3 & Concave front, biased density \\
DTLZ5 & 12 & 3 & Degenerate (curve in 3-D) \\
DTLZ6 & 22 & 3 & Degenerate, biased distance function \\
DTLZ7 & 22 & 3 & Disconnected (4 regions) \\
\midrule
WFG1  & 24 & 3 & Separable, convex, mixed, biased \\
WFG2  & 24 & 3 & Non-separable, convex, disconnected \\
WFG3  & 24 & 3 & Non-separable, linear, degenerate \\
WFG4  & 24 & 3 & Non-separable, concave, multi-modal \\
WFG5  & 24 & 3 & Non-separable, concave, deceptive \\
WFG6  & 24 & 3 & Non-separable, concave \\
WFG7  & 24 & 3 & Separable, concave, biased \\
WFG8  & 24 & 3 & Non-separable, concave, biased \\
WFG9  & 24 & 3 & Non-separable, concave, deceptive, multi-modal \\
\bottomrule
\end{tabular}
\end{table}

\subsection{Per-problem tables}
\begin{table}[t]
\centering
\scriptsize
\caption{Per-problem median normalized hypervolume. $\Delta_{\text{base}}$: AOS $-$ baseline; $\Delta_{\text{rand}}$: AOS $-$ random arm.}
\label{tab:hv_per_problem}
\setlength{\tabcolsep}{3pt}
\begin{tabular}{l|rrrrr}
\toprule
\textbf{Problem} & \textbf{Baseline} & \textbf{Random arm} & \textbf{AOS} & $\Delta_{\text{base}}$ & $\Delta_{\text{rand}}$ \\
\midrule
re21 & 0.990 & \textbf{0.991} & 0.991 & 0.001 & -0.000 \\
re24 & 0.993 & \textbf{0.993} & 0.992 & -0.000 & -0.000 \\
rwa1 & 0.994 & \textbf{0.994} & 0.994 & -0.000 & -0.000 \\
\midrule
re31 & 1.000 & \textbf{1.000} & 1.000 & 0.000 & -0.000 \\
re32 & 0.999 & 0.999 & \textbf{0.999} & 0.000 & 0.000 \\
re33 & 0.997 & \textbf{0.999} & 0.998 & 0.000 & -0.001 \\
re34 & 0.977 & 0.972 & \textbf{0.977} & 0.000 & 0.005 \\
re37 & \textbf{0.935} & 0.933 & 0.935 & -0.000 & 0.002 \\
rwa2 & 0.977 & 0.972 & \textbf{0.977} & 0.000 & 0.005 \\
rwa3 & \textbf{0.883} & 0.876 & 0.881 & -0.002 & 0.006 \\
rwa4 & 0.996 & \textbf{0.997} & 0.997 & 0.000 & -0.001 \\
rwa5 & 0.980 & 0.977 & \textbf{0.981} & 0.001 & 0.004 \\
rwa6 & \textbf{0.963} & 0.959 & 0.961 & -0.001 & 0.003 \\
rwa7 & 0.961 & \textbf{0.964} & 0.964 & 0.003 & -0.000 \\
\midrule
re41 & 0.853 & 0.774 & \textbf{0.854} & 0.001 & 0.079 \\
re42 & 0.882 & \textbf{0.885} & 0.882 & 0.000 & -0.003 \\
rwa8 & 0.947 & \textbf{0.949} & 0.947 & 0.001 & -0.002 \\
\midrule
rwa9 & \textbf{0.820} & 0.758 & 0.812 & -0.008 & 0.054 \\
rwa10 & \textbf{1.097} & 1.094 & 1.081 & -0.016 & -0.013 \\
re61 & \textbf{0.998} & 0.969 & 0.989 & -0.009 & 0.020 \\
re91 & \textbf{1.538} & 1.501 & 1.490 & -0.049 & -0.011 \\
\bottomrule
\end{tabular}
\end{table}

\begin{table}[t]
\centering
\scriptsize
\caption{Per-problem median runtime (seconds). $\Delta_{\text{base}}$: AOS $-$ baseline; $\Delta_{\text{rand}}$: AOS $-$ random arm.}
\label{tab:runtime_per_problem}
\setlength{\tabcolsep}{3pt}
\begin{tabular}{l|rrrrr}
\toprule
\textbf{Problem} & \textbf{Baseline} & \textbf{Random arm} & \textbf{AOS} & $\Delta_{\text{base}}$ & $\Delta_{\text{rand}}$ \\
\midrule
cec2009\_uf1 & \textbf{6.72} & 10.14 & 10.04 & 3.32 & -0.10 \\
cec2009\_uf2 & \textbf{7.08} & 10.73 & 10.13 & 3.05 & -0.60 \\
cec2009\_uf3 & \textbf{4.70} & 11.73 & 11.78 & 7.08 & 0.05 \\
cec2009\_uf4 & \textbf{5.48} & 10.53 & 10.74 & 5.26 & 0.21 \\
cec2009\_uf5 & \textbf{4.71} & 12.05 & 10.19 & 5.49 & -1.85 \\
cec2009\_uf6 & \textbf{4.63} & 10.03 & 10.67 & 6.04 & 0.63 \\
cec2009\_uf7 & \textbf{6.35} & 12.15 & 10.20 & 3.86 & -1.95 \\
cec2009\_uf8 & \textbf{6.63} & 15.34 & 11.56 & 4.92 & -3.79 \\
cec2009\_uf9 & \textbf{6.16} & 14.45 & 9.57 & 3.41 & -4.88 \\
cec2009\_uf10 & \textbf{6.15} & 12.73 & 9.86 & 3.70 & -2.88 \\
\midrule
lsmop1 & \textbf{6.96} & 14.65 & 10.77 & 3.80 & -3.88 \\
lsmop2 & \textbf{7.62} & 14.01 & 10.06 & 2.44 & -3.95 \\
lsmop3 & \textbf{6.50} & 11.27 & 10.68 & 4.17 & -0.59 \\
lsmop4 & \textbf{10.56} & 13.38 & 11.58 & 1.02 & -1.80 \\
lsmop5 & \textbf{6.92} & 12.68 & 8.79 & 1.87 & -3.88 \\
lsmop6 & \textbf{6.56} & 9.98 & 8.05 & 1.49 & -1.93 \\
lsmop7 & \textbf{7.17} & 11.64 & 7.99 & 0.82 & -3.65 \\
lsmop8 & \textbf{7.64} & 11.69 & 8.21 & 0.57 & -3.47 \\
lsmop9 & \textbf{6.89} & 10.84 & 7.82 & 0.93 & -3.03 \\
\bottomrule
\end{tabular}
\end{table}



\bibliographystyle{splncs04}
\bibliography{references}

\end{document}
